% Created 2019-04-28 So 09:27
% Intended LaTeX compiler: pdflatex
\documentclass[11pt]{article}
\usepackage[utf8]{inputenc}
\usepackage[T1]{fontenc}
\usepackage{graphicx}
\usepackage{grffile}
\usepackage{longtable}
\usepackage{wrapfig}
\usepackage{rotating}
\usepackage[normalem]{ulem}
\usepackage{amsmath}
\usepackage{textcomp}
\usepackage{amssymb}
\usepackage{capt-of}
\usepackage{hyperref}
\date{\today}
\title{C Learning Topics}
\hypersetup{
 pdfauthor={},
 pdftitle={C Learning Topics},
 pdfkeywords={},
 pdfsubject={},
 pdfcreator={Emacs 26.2 (Org mode 9.1.9)}, 
 pdflang={English}}
\begin{document}

\maketitle
\tableofcontents


\section{Storage classes in C}
\label{sec:org9be4cea}
[Source: geekforgeeks \href{https://www.google.com/url?client=internal-uds-cse\&cx=009682134359037907028:tj6eafkv\_be\&q=https://www.geeksforgeeks.org/storage-classes-in-c/\&sa=U\&ved=2ahUKEwi9t8CGyu\_hAhVHzaQKHcg2BqoQFjAAegQIBhAB\&usg=AOvVaw3qgAeah-xn-P8-ii93oDtq}{storage classes in C}]
Storage classes are used to describe the features of a
variable/function. These features basically include the scope,
visibility and life-time which help us to trace the existence of a
particular variable during the runtime of a program.

Syntax:
\begin{verbatim}
storage_class var_data_type var_name;
\end{verbatim}

C language uses 4 storage classes, namely:
\begin{center}
\begin{tabular}{lllll}
----------------- & ------------- & ------------- & ---------------------- & --------------------\\
Storage Specifier & Storage & Initial Value & Scope & Life\\
---------------- & ------------- & ------------- & ---------------------- & --------------------\\
auto & stack & Garbage & Within block & End of block\\
extern & Data segment & Zero & global Multiple files & Till end of program\\
static & Data segment & Zero & Within block & Till end of program\\
register & CPU register & Garbage & Within block & End of block\\
----------------- & ------------ & ------------- & ---------------------- & --------------------\\
\end{tabular}
\end{center}

\subsection{auto}
\label{sec:org3c7ee9e}
\begin{itemize}
\item This is the default storage class for all the variables declared
inside a function or block.
\item Keyword auto is generally ignored.
\item To access the auto variables outside their scope, we use pointers.
\end{itemize}
\subsection{extern}
\label{sec:org3368039}
\begin{itemize}
\item Extern storage class simply tells us that the variable is defined
elsewhere and not within the same block where it is used.
\item An extern variable is nothing but a global variable initialized with
a legal value where it is declared in order to be used elsewhere.
\item The main purpose of using extern variables is that they can be
accessed between two different files which are part of a large program.
\end{itemize}
\subsection{static}
\label{sec:org75d7905}
\begin{itemize}
\item Static variables have property of preserving the value even after
they are out of their scope.
\item The static variables are initialized once and exist till the
termination of program. Thus, no new memory is allocated because
they are not re-declared.
\end{itemize}
\subsection{register}
\label{sec:org72c49c2}
\begin{itemize}
\item This storage class declares register variables which have the same
functionality as that of the auto variables.
\item The only difference is that the  compiler tries to store these
variables in the register of the microprocessor if a free register
is available.
\item This makes the use of register variables to be much faster than that
of the variables stored in the memory during the runtime of the
program.
\item If a free register is not available, these are then stored in the
memory only. Usually few variables which are to be accessed very
frequently in a program are declared with the register keyword which
improves the running time of the program.
\item An important and interesting point to be noted here is that we
cannot obtain the address of a register variable using pointers.
\end{itemize}

\section{Volatile variables}
\label{sec:org21c4f0e}
[source: \href{https://stackoverflow.com/questions/1665452/volatile-variable\#}{volatile variable}, \href{http://msdn.microsoft.com/en-us/library/888bfst6\%2528VS.80\%2529.aspx}{type qualifier}] 
Volatile is a type qualifier not storage class specifier so it does
not determine storage location at all; it affects the definition of a
variable's type, not its storage.

It simply forces the compiler to explicitly read a variable whose type
is volatile from the variable's storage location (wherever that may
be) rather than assuming that some previously read value in a register
for example remains valid.


In C/C++, the qualifier 'volatile' indicates that an object may be
changed by something external to the program at any time and so must
be re-read from memory every time it is accessed.

\section{Macros and constants}
\label{sec:orgc3a82c0}
[Source: \href{https://stackoverflow.com/questions/3216752/what-is-the-difference-between-macro-constants-and-constant-variables-in-c}{Macros vs constants}]

Macros are handled by the pre-processor - the pre-processor does text
replacement in your source file, replacing all occurances of 'A' with
the literal 8.


Constants are handled by the compiler. They have the added benefit of
type safety.


For the actual compiled code, with any modern compiler, there should
be zero performance difference between the two.




\section{C build process}
\label{sec:org2f5b588}
[Source: \href{https://stackoverflow.com/questions/6264249/how-does-the-compilation-linking-process-work}{Stack Overflow}]

The compilation of a C++ program involves three steps:

Preprocessing: the preprocessor takes a C++ source code file and
deals with the \#includes, \#defines and other preprocessor
directives. The output of this step is a "pure" C++ file without
pre-processor directives.

Compilation: the compiler takes the pre-processor's output and
produces an object file from it.

Linking: the linker takes the object files produced by the
compiler and produces either a library or an executable file.

Preprocessing

The preprocessor handles the preprocessor directives, like \#include
and \#define. It is agnostic of the syntax of C++, which is why it must
be used with care.

It works on one C++ source file at a time by replacing \#include
directives with the content of the respective files (which is usually
just declarations), doing replacement of macros (\#define), and
selecting different portions of text depending of \#if, \#ifdef and
\#ifndef directives.

The preprocessor works on a stream of preprocessing tokens. Macro
substitution is defined as replacing tokens with other tokens (the
operator \#\# enables merging two tokens when it makes sense).

After all this, the preprocessor produces a single output that is a
stream of tokens resulting from the transformations described
above. It also adds some special markers that tell the compiler where
each line came from so that it can use those to produce sensible error
messages.

Some errors can be produced at this stage with clever use of the \#if
and \#error directives.  Compilation

The compilation step is performed on each output of the
preprocessor. The compiler parses the pure C++ source code (now
without any preprocessor directives) and converts it into assembly
code. Then invokes underlying back-end(assembler in toolchain) that
assembles that code into machine code producing actual binary file in
some format(ELF, COFF, a.out, \ldots{}). This object file contains the
compiled code (in binary form) of the symbols defined in the
input. Symbols in object files are referred to by name.

Object files can refer to symbols that are not defined. This is the
case when you use a declaration, and don't provide a definition for
it. The compiler doesn't mind this, and will happily produce the
object file as long as the source code is well-formed.

Compilers usually let you stop compilation at this point. This is very
useful because with it you can compile each source code file
separately. The advantage this provides is that you don't need to
recompile everything if you only change a single file.

The produced object files can be put in special archives called static
libraries, for easier reusing later on.

It's at this stage that "regular" compiler errors, like syntax errors
or failed overload resolution errors, are reported.  Linking

The linker is what produces the final compilation output from the
object files the compiler produced. This output can be either a shared
(or dynamic) library (and while the name is similar, they haven't got
much in common with static libraries mentioned earlier) or an
executable.

It links all the object files by replacing the references to undefined
symbols with the correct addresses. Each of these symbols can be
defined in other object files or in libraries. If they are defined in
libraries other than the standard library, you need to tell the linker
about them.

At this stage the most common errors are missing definitions or
duplicate definitions. The former means that either the definitions
don't exist (i.e. they are not written), or that the object files or
libraries where they reside were not given to the linker. The latter
is obvious: the same symbol was defined in two different object files
or libraries.


\section{Memory allocation of variables}
\label{sec:org877f187}

\section{Bit-wise operators}
\label{sec:org7722d0e}
[set-reset specific bit of a byte, swap nibbles in byte, toggling bit]

\section{Dynamic memory allocation}
\label{sec:orgab863cc}
[How does the free work? malloc, calloc and realloc; difference
between them]

\section{Pointers}
\label{sec:org65ea7e9}

\section{Call by value and call by reference}
\label{sec:orgb35f7f3}
[Write down the examples]

\section{structure padding}
\label{sec:org37d3115}
[

\section{Macros and typedefs}
\label{sec:org2abd068}
[Source: \href{https://stackoverflow.com/questions/1666353/are-typedef-and-define-the-same-in-c}{stack\(_{\text{overflow}}\)}]
'\#define' is a preprocessor token: the compiler itself will never see it.
'typedef' is a compiler token: the preprocessor does not care about
it.

typedef obeys scoping rules just like variables, whereas define stays valid until the end of the file (or until a matching undef).

Also, some things can be done with typedef that cannot be done with define.

Examples:

\begin{verbatim}
typedef int* int_p1;
int_p1 a, b, c;  // a, b, and c are all int pointers.

#define int_p2 int*
int_p2 a, b, c;  // only the first is a pointer!

.

typedef int a10[10];
a10 a, b, c; // create three 10-int arrays

.

typedef int (*func_p) (int);
func_p fp // func_p is a pointer to a function that
	  // takes an int and returns an int
\end{verbatim}

\section{Union vs structures}
\label{sec:org0f25b14}


\subsection{Pragma !?}
\label{sec:orgaebe6da}

\section{Linked lists}
\label{sec:org7484e1f}
In computer science, a Linked list is a linear collection of data
elements, whose order is not given by their physical placement in
memory. Instead, each element points to the next. It is a data
structure consisting of a collection of nodes which together represent
a sequence. In its most basic form, each node contains: data, and a
reference (in other words, a link) to the next node in the sequence. 


\section{File Scope and Global Scope: C \& C++}
\label{sec:orgc558376}


A variable with file scope can be accessed by any function or block
within a single file. To declare a file scoped variable, simply
declare a variable outside of a block (same as a global variable) but
use the static keyword.


\begin{verbatim}
static int nValue; // file scoped variable
float fValue; // global variable

int main()
{
    double dValue; // local variable
}
\end{verbatim}

File scoped variables act exactly like global variables, except their
use is restricted to the file in which they are declared.
\end{document}
